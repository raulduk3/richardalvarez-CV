%-------------------------
% Resume in Latex
% Author : Aras Gungore
% Edits: Richard A. Alvarez
% License : MIT
%------------------------

\documentclass[letterpaper,10pt]{article}

\usepackage{latexsym}
\usepackage[empty]{fullpage}
\usepackage{titlesec}
\usepackage{marvosym}
\usepackage[usenames,dvipsnames]{color}
\usepackage{verbatim}
\usepackage{enumitem}
\usepackage[hidelinks]{hyperref}
\usepackage{fancyhdr}
\usepackage[english]{babel}
\usepackage{tabularx}
\usepackage{hyphenat}
\usepackage{fontawesome5}
\usepackage{iftex}
\usepackage{fontspec}
\usepackage{xspace} 

\setmainfont{Avenir}

\frenchspacing 

%---------- FONT OPTIONS ----------
% sans-serif
% \usepackage[sfdefault]{FiraSans}
% \usepackage[sfdefault]{roboto}
% \usepackage[sfdefault]{noto-sans}
% \usepackage[default]{sourcesanspro}

% serif
% \usepackage{CormorantGaramond}
% \usepackage{charter} 

% Adjust margins
\addtolength{\oddsidemargin}{-0.5in}
\addtolength{\evensidemargin}{-0.5in}
\addtolength{\textwidth}{1in}
\addtolength{\topmargin}{-.5in}
\addtolength{\textheight}{1.0in}

\pagestyle{fancy}
\fancyhf{} % clear all header and footer fields
\fancyfoot{}
\renewcommand{\headrulewidth}{0pt}
\renewcommand{\footrulewidth}{0pt}

\urlstyle{same}

\raggedbottom
\raggedright
\setlength{\tabcolsep}{0in}

% Sections formatting
\titleformat{\section}{
	\vspace{-4pt}\scshape\raggedright\large
}{}{0em}{}[\color{black}\titlerule \vspace{-5pt}]

% Ensure that generate pdf is machine readable/ATS parsable

%-------------------------
% Custom commands

\newcommand{\xsmall}{\fontsize{8pt}{8pt}\selectfont}
\newcommand{\normal}{\fontsize{16pt}{16pt}\selectfont}
\newcommand{\titlef}{\fontsize{21pt}{21pt}\selectfont}

\newcommand{\resumeItem}[1]{
	\item\small{
		{#1 \vspace{-2pt}}
	}
}

% Footer
\fancypagestyle{firstpage}{
  \fancyhf{} % clear all header and footer fields
}

\fancypagestyle{otherpages}{
  \fancyhf{} % clear all header and footer fields

  \fancyfoot[C]{\small \\[4pt] Richard Alvarez}
}

\pagestyle{otherpages}
\thispagestyle{firstpage}

\newcommand{\resumeSubheading}[4]{
	\vspace{-2pt}\item
	\begin{tabular*}{0.97\textwidth}[t]{l@{\extracolsep{\fill}}r}
		\textbf{#1} & #2 \\
		\textit{\small#3} & \textit{\small #4} \\
	\end{tabular*}\vspace{-7pt}
}

\newcommand{\resumeSubSubheading}[2]{
	\vspace{-2pt}\item
	\begin{tabular*}{0.97\textwidth}{l@{\extracolsep{\fill}}r}
		\textit{\small#1} & \textit{\small #2} \\
	\end{tabular*}\vspace{-7pt}
}

\newcommand{\resumeEducationHeading}[6]{
	\vspace{-2pt}\item
	\begin{tabular*}{0.97\textwidth}[t]{l@{\extracolsep{\fill}}r}
		\textbf{#1} & #2 \\
		\textit{\small#3} & \textit{\small #4} \\
		\textit{\small#5} & \textit{\small #6} \\
	\end{tabular*}\vspace{-5pt}
}

\newcommand{\resumeProjectHeading}[2]{
	\vspace{-2pt}\item
	\begin{tabular*}{0.97\textwidth}{l@{\extracolsep{\fill}}r}
		\small#1 & #2 \\
	\end{tabular*}\vspace{-7pt}
}

\newcommand{\resumeOrganizationHeading}[4]{
	\vspace{-2pt}\item
	\begin{tabular*}{0.97\textwidth}[t]{l@{\extracolsep{\fill}}r}
		\textbf{#1} & {\small #2} \\
		\textit{\small#3}
	\end{tabular*}\vspace{-7pt}
}

\newcommand{\resumeSmallOrganizationHeading}[4]{
	\small{
	\vspace{-2pt}\item
	\begin{tabular*}{0.97\textwidth}[t]{l@{\extracolsep{\fill}}r}
		\textbf{#1} & {\small #2} \\
		\textit{\small#3} & \textit{\small #4} 
	\end{tabular*}\vspace{-7pt}
	}
}

\newcommand{\resumeCert}[2]{
	\small{
	\item
	\textbf{#1} {\textit{\small #2}}
	}
	\vspace{-4pt}
}

\newcommand{\resumeSubItem}[1]{\resumeItem{#1}\vspace{-4pt}}

\renewcommand\labelitemii{$\vcenter{\hbox{\tiny$\bullet$}}$}

\newcommand{\resumeSubHeadingListStart}{\begin{itemize}[leftmargin=0.15in, label={}]}
\newcommand{\resumeSubHeadingListEnd}{\end{itemize}}
\newcommand{\resumeItemListStart}{\begin{itemize}}
\newcommand{\resumeItemListEnd}{\end{itemize}\vspace{-5pt}}

\newcommand{\resumeParagraph}[1]{
	\item[]
	\parbox{\linewidth}{
		\small{#1\xspace} % Modify this line
	}
	\vspace{-7pt}
}

%-------------------------------------------
%%%%%%  RESUME STARTS HERE  %%%%%%%%%%%%%%%%%%%%%%%%%%%%

\begin{document}

%---------- HEADING ----------

\begin{center}
	\textbf{\titlef \scshape Richard Álvarez} \\ \vspace{3pt}
	\vspace{8pt}
	\small
	\hspace{.5pt} \href{mailto:rawalvarez731@gmail.com}{rawalvarez731@gmail.com} \hspace{.5pt}
	$/$
	 \hspace{.5pt} \href{https://github.com/raulduk3}{GitHub} \hspace{.5pt}
	$/$
	\hspace{.5pt} \href{https://raulduke.com}{raulduke.com} \hspace{.5pt}
	$/$
	\hspace{.5pt} \href{https://www.google.com/maps/place/Chicago,+IL/@41.833871,-87.8967704,11z/data=!3m1!4b1!4m6!3m5!1s0x880e2c3cd0f4cbed:0xafe0a6ad09c0c000!8m2!3d41.8781136!4d-87.6297982!16zL20vMDFfZDQ?entry=ttu}{Chicago, Illinois}
\end{center}

%----------- ABOUT ME  -----------

\vspace{-8pt}

\section{About Me}
\vspace{2pt}
\resumeSubHeadingListStart

\resumeParagraph{I am a researcher and web developer combining experience in filmmaking, machine learning, and visual design to create engaging digital experiences and amplify meaningful stories. I have created over 17 unique open-source projects and authored two papers with over 400 cumulative downloads. I also enjoy reading, listening to music, visiting local theaters, and cycling.}

\vspace{2pt}

\resumeSubHeadingListEnd

\vspace{-12pt}

%----------- EDUCATION -----------

\section{Education}
\vspace{4pt}
\resumeSubHeadingListStart

\resumeEducationHeading
{New York University (NYU) Tandon School of Engineering}{Brooklyn, New York}
{Master of Science in Computer Engineering}{Aug 2025}{}
{}{}

\vspace{-8pt}

\resumeEducationHeading
{Kenyon College}{Gambier, Ohio}
{Bachelor of Arts in Film; GPA: 3.4/4.0}{Aug 2020 – May 2024}
{Minor in History and Concentration in Integrated Program in Humane Studies}{}
\resumeParagraph{Authored two papers on machine learning applications in creative industries. Produced and edited over 15 experimental video projects, including music videos, audio-reactive visualizations, and short films.}

\resumeParagraph{Relevant coursework includes Senior Research Seminar (IPHS 484), AI for the Humanities (IPHS 300), Advanced Post-Production (FILM 391), Data Structures and Program Design (SCMP 218), Digital Photography (ARTS 321), Sex, Drugs, Guns: Research Strategies in the Contemporary Age (INDS 140), and Software Development (SCMP 318).}

%----------- Archived  -----------

\iffalse
\resumeSubheading
{Walter Payton College Preparatory 
% \normalfont{(Admission rate: 0.85\%)}
}{Chicago, Illinois}
{High School Diploma}{Sep 2016 \textbf{--} Jun 2020}
\fi

\resumeSubHeadingListEnd

%----------- WORK EXPERIENCE -----------

\section{Work Experience}
\vspace{4pt}
\resumeSubHeadingListStart

\resumeSubheading
{IT Assistant}{Gambier, Ohio}
{Library and Information Services (LBIS), Kenyon College · Part-time}{Sep 2023 – Feb 2024}
\resumeParagraph{I supported campus-wide technology needs by preparing workstations, moving office tech, and securely erasing and recycling equipment. I restocked printers daily. I conducted classroom checks under the guidance of team members. I streamlined team projects by applying programming skills, in one instance by generating a spreadsheet of course meetings and classroom locations to determine when our techs could perform maintenance.}

\vspace{2pt}

\resumeSubheading
{Video Editor}{Gambier, Ohio}
{Kenyon College · Contract}{Apr 2022 – May 2022}
\resumeParagraph{I condensed over 30 hours of interviews into a concise 10-minute recap for the John W. Adams Summer Scholars Program in Socio-Legal Studies. I crafted a polished visual presentation that adhered to Kenyon’s Visual Identity System. I highlighted key interview themes while ensuring balanced representation of all participants. I delivered the final product on schedule in an optimized format, maintaining professional collaboration with the employer throughout the project.}

\vspace{2pt}

\resumeSubheading
{Research Assistant}{Chicago, Illinois}
{University of Chicago · Part-time}{Aug 2018 - Nov 2019}
\resumeParagraph{I worked under Bernard Dickens III on an academic paper proposing strategies to protect against supply-chain attacks and ensure file integrity using advanced checksum technologies. I attended monthly code reviews and contributed 27 commits to the repository.}

%———– Archived  ———–
\iffalse

\resumeSubheading
{Assistant General Contractor}{Chicago, Illinois}
{Sommerlad Construction · Part-time}{Jun 2023 – Aug 2023}
\resumeParagraph{I participated in demolition projects for remodeling efforts, performed general land management tasks such as lawn mowing and simple assembly, and supported the team in meeting deadlines and adhering to safety standards.}

\resumeSubheading
{Fontanos Subs}{Chicago, Illinois}
{Sub Artist · Part-time}{May 2021 – Aug 2021}
\resumeParagraph{Worked behind the counter making sandwiches, maintaining cleanliness, and ensuring efficient service. Assisted with troubleshooting online orders, ensuring customer satisfaction and smooth operations.}

\resumeSubheading
{Private Stock Studios}{Chicago, Illinois}
{Production Assistant}{Apr 2020 – Aug 2020}
\resumeParagraph{Streamlined film and music production processes, orchestrated schedules with studio professionals, and meticulously documented production details.}

\resumeSubheading
{Intercultural Montessori}{Chicago, Illinois}
{Teacher’s Aid}{Apr 2019 – Feb 2020}
\resumeParagraph{Tutored students, facilitated programming learning, provided teacher support, and ensured effective communication and punctuality.}


\fi

\resumeSubHeadingListEnd

 %———– PUBLICATIONS ———–
\section{Publications}
\vspace{4pt}
\resumeSubHeadingListStart

\resumeProjectHeading
{\textbf{Unsupervised Deep Learning and PySceneDetect Analysis} $|$ \emph{\href{https://github.com/raulduk3/pySceneDetect-video-editing-trends}{\color{blue}GitHub}} $|$ \emph{\href{https://digital.kenyon.edu/dh_iphs_ss/22/}{\color{blue}Digital Kenyon}} $|$ \xsmall{May 23rd 2023}}{}
\vspace{1pt}
\resumeParagraph{This research focused on analyzing short-format video editing trends by leveraging PySceneDetect and unsupervised deep neural networks. Advanced data visualization techniques, including t-SNE and PCA, were employed to uncover patterns and gain insights into the editing styles and trends prevalent in the dataset.}

%———– Archived  ———–
\iffalse

\resumeProjectHeading
{\textbf{A Retrieval-Augmented Film Recommendation System} $|$ \emph{\href{https://github.com/raulduk3/language-model-driven-film-recommendation}{\color{blue}GitHub}} $|$ \emph{\href{https://digital.kenyon.edu/dh_iphs_ss/34/}{\color{blue}Digital Kenyon}} $|$ \xsmall{May 8th 2024}}{}
\vspace{1pt}
\resumeParagraph{This project utilized LangChain’s OpenAI integration to dynamically generate queries based on user preferences, showcasing the potential of advanced AI and machine learning in digital entertainment. The Retrieval-Augmented Film Recommendation System was developed using Node.js and integrated with the OMDb and TMDb APIs to enhance movie metadata, delivering precise and personalized recommendations.}

\resumeProjectHeading
{\textbf{AI-Driven Kubrick-Inspired Film Script Generation} $|$ \emph{\href{https://github.com/raulduk3/kubrick-gen}{\color{blue}GitHub}}}{}
\resumeItemListStart
\resumeItem{Created an AI pipeline for generating film scripts inspired by Stanley Kubrick’s style.}
\resumeItem{Designed an API using Dust.tt for script generation with Large Language Models.}
\resumeItemListEnd

\fi

\resumeSubHeadingListEnd

%———– CERTIFICATES ———–

\section{Certificates}
\vspace{4pt}
\resumeSubHeadingListStart

\resumeCert{CompTIA ITF+}{Sep 2024}

\resumeCert{NYU Tandon Bridge}{Mar 2025}

% \resumeCert{CompTIA A+}{Oct 2024}

% \resumeCert{CompTIA Network+}{Nov 2024}

% \resumeCert{CompTIA Security+}{Dec 2024}

\resumeSubHeadingListEnd

\iffalse
\fi

% ———– AWARDS & ACHIEVEMENTS ———–
\iffalse
\section{Awards \& Achievements}
\resumeSubHeadingListStart
\vspace{2pt}

\resumeSmallOrganizationHeading{Leadership Book Award}{Chicago, Illinois}{George Washington University}{Apr 2019}

\resumeSubHeadingListEnd
\fi

%———– SKILLS ———–
\section{Skills}
\vspace{4pt}
\resumeSubHeadingListStart

\small{\item{

\textbf{Programming}: Python, JavaScript/TypeScript, SQL, C++ \\  \vspace{2pt}
\textbf{Frameworks}: Next.js, React, Tailwind CSS, scikit-learn, LangChain \\  \vspace{2pt}
\textbf{Visualization}: After Effects, Cinema4D, Blender, Tableau, d3.js \\  \vspace{2pt}
\textbf{Open-Source Research}: Metadata analysis, web scraping, geospatial tools (Geopandas, Mapbox) \\ \vspace{2pt}


}}

\resumeSubHeadingListEnd

\newpage

\section{Websites}
\vspace{4pt}
\resumeSubHeadingListStart

\small{

\resumeSubheading
{Machine Television}{{\href{https://machinetelevision.com}{\color{blue}Visit Site}}}
{Online Store}{Oct 2024}
\resumeParagraph{Developed a functional e-commerce platform for an independent skate brand. The site was built using Next.js and Tailwind CSS for an intuitive front-end, paired with Node.js for a robust back-end infrastructure. Integrated Stripe API for seamless payment processing, optimizing user workflows across desktop and mobile. }

\vspace{4pt}

\resumeSubheading
{Joaquin Morales}{\href{https://joaquinmoralesdp.com}{\color{blue}Visit Site}}{Portfolio}{Jan 2025}
\resumeParagraph{Designed and deployed a dynamic portfolio site for a professional cinematographer. The project used Next.js for high performance, with Tailwind CSS for responsive design. Implemented a custom CMS to enable efficient content updates, managing galleries and testimonials with ease. Leveraged DigitalOcean S3 storage for scalability and fast load times for video and photo content.}

\vspace{4pt}

\resumeSubheading
{GREasyVocab Flashcards}{}{Web App}{Jul 2024}
\resumeParagraph{Created a personalized GRE vocabulary tool powered by OpenAI's APIs. The application leverages LangChain to provide personalized prompts tailored to user inputted data. Developed a secure full-stack system with user authentication and database management, ensuring a smooth and customized learning experience.}}

\resumeSubHeadingListEnd

\section{Portfolio}
\vspace{4pt}
\resumeSubHeadingListStart

\small{

\resumeSubheading
{Editor}{2022 – Present}
{Freelance}{}
\resumeItemListStart
  \resumeItem{\textit{Indie Sleaze} (2024) – Directed and produced senior thesis film, managing narrative structure and visual consistency.}
  \resumeItem{\textit{Performing the Primitive} (2023) – Reviewed footage for Sam Pack.}
  \resumeItem{\textit{Summer Legal Scholar Recaps} (2022) – Produced professional video content for Ric Sheffield, refining footage for clarity and engagement.}
\resumeItemListEnd

\vspace{6pt}

\resumeSubheading
{Music Video Producer}{2020 – Present}
{Various Collaborations}{}
\resumeItemListStart
  \resumeItem{\textit{trees} by GRAYS (2023)}
  \resumeItem{\textit{Live, Laugh, Kill} by JvneBvg (2022)}
  \resumeItem{\textit{Rick and Morty} by Black Yoshi (2021)}
  \resumeItem{\textit{Poltergeist} by 5ouley (2021) }
  \resumeItem{\textit{No Rulez} by InVoid (2021) }
  \resumeItem{\textit{Ghosts} by Undercurrent (2020) }
\resumeItemListEnd

\vspace{6pt}

\resumeSubheading
{Script Supervisor}{July 2024}
{\textit{Jasmine}}{}
\resumeParagraph{Maintained narrative and technical consistency on set.}

\resumeSubheading
{Production Assistant}{May 2024}
{\textit{Shopping for Superman}}{}
\resumeParagraph{Assisted on-set logistics and maintained production schedules.}

}

\resumeSubHeadingListEnd

\section{Additional Projects}
\vspace{4pt}
\resumeSubHeadingListStart

\resumeProjectHeading
{\textbf{A Retrieval-Augmented Film Recommendation System} $|$ \emph{\href{https://github.com/raulduk3/language-model-driven-film-recommendation}{\color{blue}GitHub}} $|$ \emph{\href{https://digital.kenyon.edu/dh_iphs_ss/34/}{\color{blue}Digital Kenyon}} $|$ \xsmall{May 8th 2024}}{}
\vspace{1pt}
\resumeParagraph{This project utilized LangChain’s OpenAI integration to dynamically generate queries based on user preferences, showcasing the potential of advanced AI and machine learning in digital entertainment. The Retrieval-Augmented Film Recommendation System was developed using Node.js and integrated with the OMDb and TMDb APIs to enhance movie metadata, delivering precise and personalized recommendations.}

\resumeProjectHeading {\textbf{AI-Driven Kubrick-Inspired Film Script Generation} $|$ \emph{\href{https://github.com/raulduk3/kubrick-gen}{\color{blue}GitHub}}}{} \resumeParagraph{Designed and developed an AI pipeline to generate film scripts inspired by Stanley Kubrick’s cinematic style. Leveraging Dust.tt and Large Language Models, I created a custom API to enable dynamic and stylistically consistent script generation. This project demonstrated the potential of generative AI for creative industries, producing scripts that emulated Kubrick's distinctive narrative and thematic characteristics.}

\resumeProjectHeading {\textbf{Sentiment Analysis of Rotten Tomatoes Reviews} $|$ \emph{\href{https://github.com/raulduk3/rotten-tomatoes-sentiment}{\color{blue}GitHub}}}{} \resumeParagraph{Conducted a sentiment analysis of user reviews from Rotten Tomatoes to evaluate the psychology of movie consumers and the reliability of the platform's ‘freshness’ indicators. Using VADER and the NLTK library, I processed text data by filtering stop words, tokenizing reviews, and extracting sentiment scores. The analysis revealed that 73 of user reviews aligned with their assigned ‘freshness’ labels, validating both the reliability of Rotten Tomatoes user ratings and the effectiveness of the VADER sentiment analysis tool. Findings were presented in detailed visual reports, highlighting correlations between sentiment and user ratings.}

\resumeSubHeadingListEnd

\newpage

\section{Project Proposal}
\vspace{4pt}
\resumeSubHeadingListStart

\resumeSubheading
{\textbf{Mapping Migration: Open-Source Visual Investigations into Refugee Integration}}{}{Proposed Project for SITU Research}{}
\resumeParagraph{This project explores the migration and integration patterns of refugees from Ukraine and Venezuela in Chicago, focusing on disparities in placement, access to city infrastructure, and federal enforcement actions. It leverages open-source investigation techniques to examine how refugees transition into urban systems, tracking ICE activity and identifying patterns of enforcement.}

\vspace{4pt}

\resumeParagraph{\textbf{Key Methods and Tools}}
\vspace{-8pt}
\resumeItemListStart
\resumeItem{Web scraping public datasets (e.g., news outlets, city reports, ICE records) using Python libraries like BeautifulSoup and Scrapy, combined with natural language processing (NLP) to extract key insights from text data. This enables the rapid aggregation of information about refugee demographics, resource availability, and enforcement patterns.}
\resumeItem{Using advanced NLP techniques with LLMs, including named entity recognition (NER) and sentiment analysis, to identify geographic locations, trends in refugee experiences, and policy themes from large datasets. For example, parsing city council minutes or deportation notices to detect disparities in resource distribution or enforcement practices.}
\resumeItem{Building machine learning pipelines with scikit-learn for clustering refugees based on attributes such as housing access or economic integration. These models can help detect systemic patterns that would otherwise remain hidden in raw data.}
\resumeItem{Mapping refugee placement and ICE operations with foundational geospatial tools like Mapbox or Python’s Geopandas. By overlaying these datasets, we can identify areas of concentrated enforcement activity or insufficient infrastructure, providing actionable insights.}
\resumeItem{Developing interactive, web-based dashboards using tools like three.js or d3.js to visually represent data trends. For instance, creating timelines of enforcement events or heatmaps of refugee settlement to highlight correlations and inequities.}
\resumeItem{Integrating metadata analysis and reverse image search techniques to verify multimedia sources, enhancing data reliability and accountability when analyzing sensitive content such as deportation footage or site-specific evidence.}
\resumeItemListEnd

\resumeParagraph{These processes provide a granular understanding of refugee placement, infrastructure access, and enforcement targeting. By connecting spatial and contextual data, the project uncovers patterns critical for advocacy, legal challenges, and policy reform—offering tools and insights beyond standard geospatial or investigative approaches.}



\vspace{4pt}

\resumeParagraph{\textbf{Impact:} This project aligns with SITU Research’s mission by using cutting-edge tools to amplify truth, promote equity, and address urgent human rights issues. It reflects a deep curiosity for mastering new technologies while demonstrating proficiency in integrating data science, geospatial analysis, and visual storytelling to support advocacy efforts.}

\resumeSubHeadingListEnd


\newpage

%----------- REFERENCES -----------	
\section{References}
\vspace{2pt}
\resumeSubHeadingListStart
\vspace{2pt}

\item{
\textbf{Steve Krajenski} \\ 
{ LBIS Kenyon College \\  
\href{mailto:krajenski1@kenyon.edu}{krajenski1@kenyon.edu}} \\
(740) 504-0328 \\
M - F: 1 PM to 3 PM ET 
}

\item{
\textbf{Jon Chun} \\ 
{ Visiting Instructor at Kenyon College \\  
\href{mailto:chunj@kenyon.edu}{chunj@kenyon.edu}} \\
}

\item{
\textbf{Bernard Dickens III} \\ 
{ Ergodark LLC \\ 
\href{mailto:me@bernarddickens.com}{me@bernarddickens.com}} \\ 
} 

\item{
\textbf{Audra Anderson} \\ 
{ BDPA Mentor \\  
\href{mailto:ajagwwtravels@gmail.com}{ajagwwtravels@gmail.com}} \\
}

\item{
\textbf{Tina Vazquez} \\ 
{ Fontano's Sub Sandwiches \\  
(414) 232-1728 \\
M - F: 9 AM to 3 PM CT 
}
}


\resumeSubHeadingListEnd

%————— END ————————––

\end{document}